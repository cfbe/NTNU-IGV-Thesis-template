\chapter{Background}
\label{chap:Background}

The background section should give sufficient background information to give the reader an overview of the state-of-the-art research field in which you work.

The background chapter can be a bit similar to the introduction. For journal articles, it is not that common with a background section, as it is not kept separate from the introduction. However, for a master thesis, it is common to have more background and therefore to separate out some of the background material into a separate chapter. For a specialization project, the background chapter could be the most important chapter, as the specialization project is centered around learning a new subject field.

Do not overestimate the background knowledge of the reader. As a rule of thumb, you could assume (s)he knows as much as you did when you started on your master thesis/project. So you need to give the reader enough background to be able to read the rest of your thesis.

All relevant literature should be included, so do a thorough literature search. When you do, try not to miss out on the classics and defining papers in your field of work. For a specialization project, your background section can include references to display the breadth of the field you are working on. In contrast, for a master thesis, you should be able to argue for all included references. Do not add references just to get a long reference list. After you have spent a lot of time reading something, it could be tempting to add those references to just show off that you have read them, even though it turned out after reading them that it is not fully related to your work. Do not, all included references in your master thesis should be relevant to your work.

The background material should also motivate your work. It should make your research question interesting by showing what others have done, and showing what is missing, highlighting the void that you try to fill with your work.


\section{Citations}

The background needs relevant literature to place your project work into context. The Google-scholar search (\url{scholar.google.com}) is a good starting point for searching for relevant papers.  This subsection will show how to include these references in your \LaTeX-document.

There is a long range of different styles and packages in Latex for citations. During your writing process, it is often beneficial to have an \texttt{authoryear} style, where you see the author(s) and the year of publication. This will help you remember what the reference is. In the \texttt{main.tex} file you will find a command defining the bibliography style:
\lstinputlisting[firstline=39, lastline=52]{main.tex}

This is where you want to go to change the style of your referencing. In this setup, we use the \texttt{natbib} style. This allows for using \verb=\citep= and \verb=\citet= references, which are useful if you use \texttt{authoryear} style:
\begin{itemize}
    \item Whenever the reference is part of the sentence, you should use the textual citation \verb=\citet=. The \verb=\citet= reference types will give a reference that looks like this: \citet{berg2014permeability}.
    \item Whenever the reference is not a part of the sentence, but just general for the sentence or paragraph, you should use the parenthetical citation \verb=\citep=. The \verb=\citep= reference types will give a reference that looks like this: \citep{berg2014permeability}.
    \item If you do not specify, but use \verb=\cite=, it will look like this: \cite{berg2014permeability}.
\end{itemize}

All the references you use will automatically show up in your reference list. If you want to shift to numerical citations, you should use the \texttt{cite} command, as there are no differences between textual and parenthetical citations when you use the numerical style.