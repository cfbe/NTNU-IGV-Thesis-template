\chapter{Methods}

In this chapter, you describe the methods used to obtain your results. This can be new methods for analyzing existing data, or existing methods applied to new data. Include a complete description of your methods. It is often helpful to use flow charts to explain your methods.

Give enough details for others to be able to reproduce your work. Please do not overdo it, as a rule of thumb you can assume that the reader has the same background as you had when you started the thesis.

Introduce and describe all parameters that have been tested in your project, and why these in particular have been varied. What was the reason for varying these parameters?

Additional details can be given in the appendices. Appendices are useful to avoid too much information in the thesis itself, which can be detrimental to the reading experience.

If you are developing or using software, then it is common to include pseudo-codes for the software. The full code can be added as an appendix, but it is even better to upload the code to a public repository (e.g., \url{github.com}), and link the repository from the appendix.

\section{Writing tips}

Here are some general writing tips for best practices in scientific writing.

\begin{itemize}
    \item Use short sentences. Shorter sentences are easier for a reader to follow than longer sentences. As a rule of thumb, if you can split a sentence in two, it is usually for the better.
    \item Try to avoid adverbs and adjectives, e.g., quickly, slowly, densely, etc., as they are often not quantitative. A densely populated area in Finmark is not the same as a densely populated area in China.
    \begin{itemize}
        \item Avoid: The reaction completed quickly.
        \item Use: The reaction completed in 5 minutes.
    \end{itemize}
    \item If no one did the action, then it is sometimes common to avoid the passive voice. The active voice is usually clearer and more concise than the passive.
    \begin{itemize}
        \item Passive voice: Experiments were run.
        \item Active voice: We ran experiments.
    \end{itemize}
    \item Minimize the use of new abbreviations. As a reader, it is difficult to remember the meaning of more than a handful of new abbreviations. Commonly used symbols, such as $\phi$ for porosity, are an exception. If you do introduce numerous new abbreviations, it is useful to add a list of abbreviations at the beginning of your document (an example is included at the beginning of this document).
    \item A noun should follow \emph{this}, \emph{these}, etc.
    \begin{itemize}
        \item Avoid: This shows \dots
        \item Use: This result shows \dots
    \end{itemize}
    \item Avoid using \emph{it}, instead, try to be specific.
    \begin{itemize}
        \item Avoid: It revealed the importance of science.
        \item Use: The experiment revealed the importance of science.
    \end{itemize}
    \item Avoid "to be" verbs. Sentences with these verbs can almost always be shortened.
    \begin{itemize}
        \item Avoid: The reaction rate was decreasing.
        \item Use: The reaction rate decreased.
    \end{itemize}
    \item Be careful about verb tenses. Did something actually happen in the past, or are you presenting a result or observation that is still true today? Papers with numerical models are often written in the present tense, while papers focusing on experiments are often written in the past tense. But even in experimental papers, some sentences can be in the present tense if they refer to things that are still true, i.e., "The data shows \dots"
    \item Avoid putting words in quotations. If the idea or object needs quotations around something that describes it, then this usually means that you can find a better way to describe it.
    \begin{itemize}
        \item Avoid: The experiment modeled "serpentinization".
        \item Use: The experiment captured key aspects of the serpentinization process. 
    \end{itemize}
    \item Comma rules; \emph{that} does not follow a comma, whereas \emph{which} should follow a comma.
    \begin{itemize}
        \item Incorrect: The data, that we collected \dots
        \item Correct: The data that we collected \dots
        \item Incorrect: The students were angry which fueled the protests.
        \item Correct: The students were angry, which fueled the protests.
    \end{itemize}
    \item Contractions are common in speech and informal writing, but it is not used in formal writing.
    \begin{itemize}
        \item Use \emph{it is}, not \emph{it's}.
        \item Use \emph{does not}, not \emph{doesn't}.
        \item Use \emph{is not}, not \emph{isn't}.
        \item Use \emph{we will}, not \emph{we'll}.
    \end{itemize} 
    \item Some words are synonymous with others, but less common in formal scientific writing.
    \begin{itemize}
        \item Use \emph{although} instead of \emph{though}.
        \item Use \emph{in addition} or \emph{moreover} instead of \emph{besides}.
        \item Use \emph{however} instead of \emph{but}.
    \end{itemize}
\end{itemize}