\chapter{Introduction}
\label{chap:Introduction}

In your introduction, it is common to start with a broad focus and then narrow in. You could narrow it even further in the background section. The introduction section should give the perspective and background for your upcoming research question. This is where you can take an expansive and holistic view, before focusing on your question.

\section{Motivation}

It is common to have a motivation section in your introduction (it does not need to be singled out as a separate section, though; your motivation could be just a part of your introduction). In this section, you should motivate your research question, which will come later. Why is your work interesting? Why is it important? Why is it necessary? What is the purpose and aim of this work? This motivation should lead to the research question being asked later.

In this section, you should also distinguish your work from earlier studies. In the introduction, you should give a very brief overview of the current state of the art in the research field, and a more throughout discussion on earlier literature should be provided in the background chapter (Chapter \ref{chap:Background}). In this section, try to briefly point out what has already been done, what the earlier research showed, what the limitations of earlier research were, and how your work will address these shortcomings.


\subsection{Research question}

The research question you are trying to answer through your thesis should be formulated in the introduction section. The purpose of the motivation should lead to a hypothesis, and the research questions should be formulated so that they can verify the hypothesis.

You can also formulate a set of objectives for your work to answer your research question.  It should be clear how these objectives together will answer your research question.

\section{Outline}

It is common to end the introduction with an outline of the thesis. Here, you could briefly present the different chapters and their content.